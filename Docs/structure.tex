\usepackage{amsmath, amsfonts, amsthm} % Math packages

\usepackage{listings} % Code listings, with syntax highlighting

\usepackage[utf8]{inputenc} % Required for inputting international characters
\usepackage[T1]{fontenc} % Use 8-bit encoding

\usepackage[english, greek]{babel} % English language hyphenation

\usepackage{graphicx} % Required for inserting images
\graphicspath{{Figures/}{./}} % Specifies where to look for included images (trailing slash required)

\usepackage{booktabs} % Required for better horizontal rules in tables

\usepackage{dirtytalk} % Required for quoting.

\usepackage{float} % Added for hard placement of images.

\usepackage[dvipsnames]{xcolor} % Added for extra colors.

\usepackage{tikz} % For colored boxes and more.

\numberwithin{equation}{section} % Number equations within sections (i.e. 1.1, 1.2, 2.1, 2.2 instead of 1, 2, 3, 4)
\numberwithin{figure}{section} % Number figures within sections (i.e. 1.1, 1.2, 2.1, 2.2 instead of 1, 2, 3, 4)
\numberwithin{table}{section} % Number tables within sections (i.e. 1.1, 1.2, 2.1, 2.2 instead of 1, 2, 3, 4)

\usepackage{translator}

\newcommand{\en}[1]{\foreignlanguage{english}{#1}}
\newcommand{\src}[1]{\texttt{\en{#1}}}


% Extra Formatting

\setlength{\parindent}{0em}
\setlength{\parskip}{0em}


% Code Listing Style

\lstdefinestyle{code}{
  belowcaptionskip=1\baselineskip,
  breaklines=true,
  frame=LRTB,
  xleftmargin=\parindent,
  showstringspaces=false,
  basicstyle=\ttfamily,
  keywordstyle=\bfseries\color{green!40!black},
  commentstyle=\itshape\color{purple!40!black},
  identifierstyle=\color{black},
  stringstyle=\color{orange},
}


\newcommand{\lstcode}[3]
{
    \begin{otherlanguage}{english}
    \lstinputlisting[language=#2, frame=single, style=code, caption=#3]{#1}
    \end{otherlanguage}
}

% Some Greek stuff
\usepackage{graphicx}
\usepackage{indentfirst}
\usepackage{verbatim}
\usepackage{amsmath}
\usepackage{amsthm}
\usepackage{amssymb}
\usepackage{hyphenat}
\usepackage{makeidx}

\addto\captionsgreek{%
  \renewcommand{\indexname}{Ευρετήριο όρων}%
  \renewcommand{\bibname}{\textgreek{Βιβλιογραφία}}%
}

% latin text (and greek text)
\newcommand{\tl}[1]{\textlatin{#1}}
\newcommand{\tg}[1]{\textgreek{#1}}
\newcommand{\lat} {\latintext} 
\newcommand{\gre}{\greektext}
